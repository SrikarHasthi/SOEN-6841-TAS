\section{References}

    \begin{enumerate}[label={[\arabic*]}]
    
        
        \item Judge, T. A., Bono, J. E., Ilies, R., \& Gerhardt, M. W. (2002). The conscientiousness-performance relationship: A meta-analysis. Personnel Psychology, 55(3), 797-82 \url{http://www.timothy-judge.com/Judge,%20Bono,%20Ilies,%20&%20Gerhardt.pdf}

        \item Myers, I. B., \& Briggs Myers, K. (1962). Manual: The Myers-Briggs type indicator. Princeton, NJ: Educational Testing Service \url{https://www.semanticscholar.org/paper/The-Myers-Briggs-Type-Indicator%3A-Manual-(1962).-Myers/d39f44fd94ab4e73a826aa1e4c0fcbc49377838d}

        \item \label{itm:third} Cohen, Y., Ornoy, H., \& Keren, B. (2013). Personality types of project managers and their success: A field survey. Psychology \& Marketing, 30(11), 927-935. \url{https://www.researchgate.net/publication/263596981_MBTI_Personality_Types_of_Project_Managers_and_Their_Success_A_Field_Survey}

        \item \label{itm:fourth} Hussain, A., Jamil, M., Farooq, M. U., Asim, M., Rafique, M. Z., \& Pruncu, C. (2021). Project Managers’ Personality and Project Success: Moderating Role of External Environmental Factors. Sustainability, 13(16), 9477.\url{https://www.researchgate.net/publication/354094465_Project_Managers'_Personality_and_Project_Success_Moderating_Role_of_External_Environmental_Factors}

        \item \label{itm:fifth} Ullah, S., Khan, N. A., \& Farooq, S. (2022). The impact of manager's personality traits on project success through affective professional commitment: the moderating role of organizational project management maturity system. International Journal of Project Management, 40(4), 352-365 \url{https://www.mdpi.com/2071-1050/13/16/9477}

        \item \label{itm:sixth} Creswell, J. W. (2014). Research design: Qualitative, quantitative, and mixed methods approaches (4th ed.). Thousand Oaks, CA: Sage Publications. \url{https://www.ucg.ac.me/skladiste/blog_609332/objava_105202/fajlovi/Creswell.pdf}

        \item \label{itm:seventh} Braun, V., \& Clarke, V. (2006). Using thematic analysis in psychology. Qualitative Research in Psychology, 3(2), 77-101.\url{https://www.researchgate.net/publication/235356393_Using_thematic_analysis_in_psychology}

         \item \label{itm:eight}Morgeson, F. P., DeRue, D. S., \& Karam, E. P. (2010). Leadership in Teams: A Functional Approach to Understanding Leadership Structures and Processes. Journal of Management, 36(1), 5-39.\url{https://journals.sagepub.com/doi/abs/10.1177/0149206309347376}

         \item \label{itm:ninth}Locke, E. A., \& Latham, G. P. (2002). Building a Practically Useful Theory of Goal Setting and Task Motivation: A 35-Year Odyssey. American Psychologist, 57(9), 705-717 \url{https://psycnet.apa.org/record/2002-15790-003}

         \item \label{itm:tenth}Morgeson, F. P., DeRue, D. S., \& Karam, E. P. (2010). Leadership in Teams: A Functional Approach to Understanding Leadership Structures and Processes. Journal of Management, 36(1), 5-39. \url{https://journals.sagepub.com/doi/abs/10.1177/0149206309347376}
  
    \end{enumerate}