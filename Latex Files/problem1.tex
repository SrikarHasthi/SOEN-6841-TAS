
\section{Introduction}
\vspace{1\baselineskip}

\subsection{Motivation}

The motivation for this research is to investigate the relationship between personality type and project management success. Project management is a complex and challenging field, and the success of a project often depends on the ability of the project manager to effectively lead and motivate a team.

There is a growing body of research on the relationship between personality type and job performance in general, but there is less research on the relationship between personality type and project management success specifically. The need to explore how specific personality traits align with various project requirements is vital, as it directly impacts project outcomes and team dynamics. This is a gap in the literature that this research aims to address.

\subsection{Problem Statement}
The specific problem statement being investigated in this research is:

How does personality type influence project management success?

The investigation focuses on the precise examination of how different personality types of influence project management outcomes. With projects becoming more intricate, diverse, and team-oriented, it is imperative to identify the specific traits that lead to successful project delivery. The problem lies in the challenge of pinpointing these traits accurately and understanding their nuanced impact on project management processes. Addressing this challenge is crucial for businesses and organizations striving to optimize their project teams and ensure efficient project execution.

\subsection{Objectives}
The objectives of this research are to:
\begin{enumerate}
    \item Identify the key personality traits that are associated with project management success.
    \item Develop a model that can predict project management success based on personality type.
    \item Provide recommendations for how project managers can improve their success by developing their key personality traits.
\end{enumerate}
This research will benefit both project managers and organizations that employ project managers. By understanding the relationship between personality type and project management success, project managers can develop their key personality traits and organizations can select the right project managers for their projects. The ultimate goal of this research is to improve project teams' productivity and efficiency in a variety of industries, which will benefit people and companies alike.




















\section* {Deliverable 1}
2nd October 2023

    \subsection*{PROBLEM 1 [10 MARKS]}

    Using the Goal-Question-Metric (GQM) approach (or one of its extensions), present one goal specific to METRICSTICS and articulate 2N questions related to that goal, where N is the team size. Discuss whether any metrics help answer those questions. 
    \\\\
    NOTES
    \\
    The goals must aim to be SMART.

    \subsection*{PROBLEM 2 [30 MARKS]} 
    Using the given description, construct a use case model for METRICSTICS. This construction must be accompanied by definitions of actors and use cases, and descriptions of use case scenarios.
    \\\\
    NOTES
    \\
    There can be several use cases, including carrying out a sequence of calculations, saving data in memory, restarting a session, and so on. A statistical calculator1 could be used as a motivation to `elicit' necessary use cases.

\pagebreak

\section {GQM for Descriptive Statistics}
    Using the Goal-Question-Metric (GQM) approach (or one of its extensions), present one goal specific to DESCRIPTIVE-STATISTICS and articulate 2N questions related to that goal, where N is the team size. Discuss whether any metrics help answer those questions.\\ Note. The goals must aim to be SMART.
  
    \subsection{Goal}
    Following the template given in the lecture notes {[}1{]}:
    
    \strut \\
    \textbf{\large Purpose}
    \strut \\ \\
    To \emph{develop within the given time limit from
    26th September 2023 to 24th
    November 2023}, the \emph{artifacts of a comprehensive system
    capable of taking a random set of data values as input} and
    generating the following descriptive statistics:\\
    \strut \\
    1. Minimum (m),\\
    2. Maximum (M),\\
    3. Mode (o),\\
    4. Median (d),\\
    5. Mean absolute deviation (MAD), and\\
    6. Standard deviation ($\sigma$).\\\\
    in order to \emph{provide a numerical description of an arbitrary
    set of data values.
    }
    
    \strut \\
    \textbf{\large Perspective}
    \strut \\
    \strut \\
    Examine \emph{the accuracy of Descriptive Statistics from a
    technical user\textquotesingle s (researchers, analysts, and data
    scientists) perspective.
    }
    
    \strut \\
    \textbf{\large Environment}
    \strut \\
    \strut \\
    In the context of the \emph{development and testing phase of a project\textquotesingle s lifecycle.
    }

    
    \pagebreak

    
    \section {The goal in terms of a SMART Goal:}
    
    \strut \\
    \textbf{\large Specific}
    \strut \\
    \strut \\
    By the end of the project duration, we would have developed the
    METRICSTICS Descriptive Statistics System, which will accurately
    calculate key statistical measures (minimum, maximum, mode, median,
    mean absolute deviation, and standard deviation) for input datasets,
    both real and artificially generated, through an intuitive user
    interface.
    
    \strut \\
    \textbf{\large Measurable}
    \strut \\
    \strut \\
    We will measure the success of this goal by using metrics mentioned in section 1.2
    
    \strut \\
    \textbf{\large Achievable}
    \strut \\
    \strut \\
    To achieve this goal, we will allocate adequate resources, including a dedicated development team, a testing resource, and project management oversight.
    
    \strut \\
    \textbf{\large Relevant}
    \strut \\
    \strut \\
    Developing the METRICSTICS Descriptive Statistics System is directly aligned with our team's mission to address the growing need for efficient and accurate descriptive statistical analysis, essential for technical users such as researchers, analysts, and data scientists across various industries.
    
    \strut \\
    \textbf{\large Time-Bound}
    \strut \\
    \strut \\
    We will complete the development, testing, release, and documentation of METRICSTICS within the next 3 months. This timeline allows for comprehensive development, testing, refinement, and documentation while ensuring timely delivery to meet user needs.
    \strut \\
    \strut \\
    By adhering to this SMART goal, we aim to deliver a valuable tool that enhances the capabilities of technical users to perform descriptive statistical analysis quickly, accurately, and efficiently.
    
    \pagebreak

    \section{Questions and Metrics}
    
    \begin{enumerate}
        \item Is it possible to complete the project within the allotted time limit?
        \begin{itemize}
            \item M1: Task completion rate.
            \item M2: Burn down chart.
            \item M3: Leadtime {[}2{]} (time required to go from idea to delivered software)
        \end{itemize}

        \item Does the existing functionality align with the expectations of a technical user?
        \begin{itemize}
            \item M4: Technical User\textquotesingle s Heuristic Assessment
        \end{itemize}

        \item What are the program\textquotesingle s dimensions?
        \begin{itemize}
            \item M6: Storage capacity needed to run the software application.
            \item M7: Lines of Code (LOC)
        \end{itemize}

        \item Is the code efficient and well-optimized?
        \begin{itemize}
            \item M8: Number of duplicated Lines of Code
            \item M9: The computational time complexity of functions
        \end{itemize}

        \item Is the program easy to maintain?
        \begin{itemize}
            \item M10: Documentation {[}3{]}
            \item M11: Maintainability Index {[}4{]}
        \end{itemize}

        \item How efficient is the team's performance?
        \begin{itemize}
            \item M12: Velocity
            \item M13: Average Cycle Time {[}5{]}
        \end{itemize}

        \item What is the level of satisfaction among the project\textquotesingle s stakeholders?
        \begin{itemize}
            \item M14: Team Satisfaction
            \item M15: Customer Satisfaction
        \end{itemize}

        \item Do all the functions produce the expected output?
        \begin{itemize}
            \item M16: Unit test code coverage
        \end{itemize}

        \item Are there any defects during the development?
        \begin{itemize}
            \item M17: Bugs and defect count
        \end{itemize}

        \item  Is the project equipped to handle errors and invalid inputs during its execution?
         \begin{itemize}
            \item M18: Robustness Index {[}6{]}
            \item M19: Function robustness {[}6{]}
        \end{itemize}

        \item What are the user stories' priority?
         \begin{itemize}
            \item M20: User Story Point
        \end{itemize}

        \item Is the code designed to be independent of any specific IDE
        (Integrated Development Environment)?
        \begin{itemize}
            \item M21:  Portability {[}7{]}
        \end{itemize}
        
    \end{enumerate}
    
    \pagebreak
    \section{Use Case Model}
    Using the given description, construct a use case model for DESCRIPTIVE STATISTICS.\\ Note. There can be several use cases, including saving data in memory, restarting a session, and so on. A statistical calculator could be used as a motivation to `elicit' necessary use cases.
    
    \subsection{Use Case Model Diagram}
    Following the use case templates {[}8{]} and use case diagram example {[}9{]} we get the following:
    
    \begin{center}
    \includegraphics[width=5.5in,height=6.5in]{METRICSTICS/media/UseCaseDiagram.png}
    \end{center}
    
    \begin{center}
    Figure 1: Use case model diagram for Descriptive Statistic
    \end{center}
    
    \strut \\
    \textbf{\large Actors}
    \strut \\
    The actor is a technical user who can be a researchers, analysts, student or data scientists. They will use the system to obtain descriptive statistics for research, building models or assignment.
    
    \pagebreak
    
    \subsection{Use Case Model Tables }
    \strut \\
    \textbf{\large Use Case 1}

    \begin{longtable}{|p{0.227\linewidth}|p{0.773\linewidth}|}
    \hline
    \textbf{Title} & \textbf{Input data set} \\
    \hline
    \endfirsthead
    \hline
    \textbf{Title} & \textbf{Input data set} \\
    \hline
    \endhead
    \hline
    \endfoot
    \hline
    \endlastfoot
    Description & Provide a finite data set x\textsubscript{1},
    x\textsubscript{2}, x\textsubscript{3}, \ldots. x\textsubscript{n} as
    input from one of the below two sources:
    
    1. Keyboard,
    
    2. File and
    
    3. Autogenerated. \\ \hline
    Primary Actor & Technical User \\ \hline
    Preconditions & 1. The program must be running
    
    2. The input source must be in proper comma separated values format
    
    3. The input should be numbers
    
    4. There should be at least 1 value in the input file \\ \hline
    Success Guarantee & Data set is assigned to a variable \\ \hline
    Main Success Scenario & 1. User inputs a valid dataset
    
    2. The system sorts the data in an ascending order \\
    Extensions & An error message is showed indicating the data set given
    was invalid or in the wrong format.
    \end{longtable}
    \strut \\
    \textbf{\large Use Case 2}

    \begin{longtable}{|p{0.227\linewidth}|p{0.773\linewidth}|}
    \hline
    \textbf{Title} & \textbf{Get Minimum} \\
    \hline
    \endfirsthead
    \hline
    \textbf{Title} & \textbf{Get Minimum} \\
    \hline
    \endhead
    \hline
    \endfoot
    \hline
    \endlastfoot
    Description & Calculate the minimum value from the given data set. \\ \hline
    Primary Actor & Technical User \\ \hline
    Preconditions & 1. The system must be running.
    
    2. A valid data set must have been inputted into the system. \\ \hline
    Success Guarantee & The minimum value is successfully calculated and
    displayed. \\ \hline
    Main Success Scenario & 1. The system retrieves the sorted data set.
    
    2. The system identifies the smallest value in the data set as the
    minimum.
    
    3. The system displays the calculated minimum value.
    
    4. The user returns to the main menu of operations. \\
    \hline
    Extensions & If the data set is empty, an error message indicating that
    the input is empty is displayed, and the user returns to the main menu
    without minimum value calculation.
    \end{longtable}
    \strut \\
    \textbf{\large Use Case 3}
    
    \begin{longtable}{|p{0.227\linewidth}|p{0.773\linewidth}|}
    \hline
    \textbf{Title} & \textbf{Get Maximum} \\
    \hline
    \endfirsthead
    \hline
    \textbf{Title} & \textbf{Get Maximum} \\
    \hline
    \endhead
    \hline
    \endfoot
    \hline
    \endlastfoot
    Description & Calculate the maximum value from the given data set. \\ \hline
    Primary Actor & Technical User \\ \hline
    Preconditions & 1. The system must be running.
    
    2. A valid data set must have been inputted into the system. \\ \hline
    Success Guarantee & The maximum value is successfully calculated and
    displayed. \\ \hline
    Main Success Scenario & 1. The system retrieves the sorted data set.
    
    2. The system identifies the largest value in the data set as the
    maximum.
    
    3. The system displays the calculated maximum value.
    
    4. The user returns to the main menu of operations. \\ \hline
    Extensions & If the data set is empty, an error message indicating that
    the input is empty is displayed, and the user returns to the main menu
    without maximum value calculation.
    \end{longtable}
    
    \pagebreak
    
    \strut \\
    \textbf{\large Use Case 4}
    
    \begin{longtable}{|p{0.227\linewidth}|p{0.773\linewidth}|}
    \hline
    \textbf{Title} & \textbf{Get Mode Value} \\
    \hline
    \endfirsthead
    \hline
    \textbf{Title} & \textbf{Get Mode Value} \\
    \hline
    \endhead
    \hline
    \endfoot
    \hline
    \endlastfoot
    Description & Calculate the mode value(s) from the given data set which
    is the most frequently appeared value in the data set (mode value does
    not have to be unique). \\ \hline
    Primary Actor & Technical User \\ \hline
    Preconditions & 1. The system must be running.
    
    2. A valid data set must have been inputted into the system. \\ \hline
    Success Guarantee & The mode value(s) is/are successfully displayed. \\ \hline
    Main Success Scenario & 1. The system retrieves the sorted data set.
    
    2. The system analyzes the data set to identify the value(s) that
    appears most frequently.
    
    3. The identified value(s) is designated as the mode value(s).
    
    4. The system displays the calculated mode value(s).
    
    5. The user returns to the main menu of operations. \\ \hline
    Extensions & If the data set is empty, an error message indicating that
    the input is empty is displayed, and the user returns to the main menu
    without mode value(s) calculation.
    \end{longtable}

    \strut \\
    \textbf{\large Use Case 5}
    
    \begin{longtable}{|p{0.227\linewidth}|p{0.773\linewidth}|}
    \hline
    \textbf{Title} & \textbf{Get Median Value} \\
    \hline
    \endfirsthead
    \hline
    \textbf{Title} & \textbf{Get Median Value} \\
    \hline
    \endhead
    \hline
    \endfoot
    \hline
    \endlastfoot
    Description & Calculate the median value from the given data set. The
    median is the middle number if the number of data values is odd, and it
    is the arithmetic mean of the two middle numbers if number of data
    values is even. \\ \hline
    Primary Actor & Technical User \\ \hline
    Preconditions & 1. The system must be running.
    
    2. A valid data set must have been inputted into the system. \\ \hline
    Success Guarantee & The median value is successfully calculated and
    displayed. \\ \hline
    Main Success Scenario & 1. The system retrieves the sorted data set.
    
    2. The system checks the number of data values (n) in the dataset.
    
    3. If n is odd:
    
    The system identifies the middle value as the median.
    
    If n is even:
    
    The system calculates the arithmetic mean of the two middle values,
    which is designated as the median.
    
    4. The system displays the calculated median value.
    
    5. The user returns to the main menu of operations. \\ \hline
    Extensions & If the data set is empty, an error message indicating that
    the input is empty is displayed, and the user returns to the main menu
    without median value calculation.
    \end{longtable}
    
    \pagebreak

    \strut \\
    \textbf{\large Use Case 6}
    
    \begin{longtable}{|p{0.227\linewidth}|p{0.773\linewidth}|}
    \hline
    \textbf{Title} & \textbf{Get Arithmetic Mean Value} \\
    \hline
    \endfirsthead
    \hline
    \textbf{Title} & \textbf{Get Arithmetic Mean Value} \\
    \hline
    \endhead
    \hline
    \endfoot
    \hline
    \endlastfoot
    Description & Calculate the arithmetic mean value from the given data
    set.
    
    \includegraphics[width=0.97508in,height=0.70839in]{METRICSTICS/media/Mean.png} \\ \hline
    Primary Actor & Technical User \\ \hline
    Preconditions & 1. The system must be running.
    
    2. A valid data set must have been inputted into the system. \\ \hline
    Success Guarantee & The Arithmetic Mean value is successfully calculated
    and displayed. \\ \hline
    Main Success Scenario & 1. The system retrieves the sorted data set.
    
    2. The system calculates the arithmetic mean by summing all the values
    in the data set and dividing by the number of data values.
    
    3. The system displays the calculated arithmetic mean value.
    
    4. The user returns to the main menu of operations. \\ \hline
    Extensions & If the data set is empty, an error message indicating that
    the input is empty is displayed, and the user returns to the main menu
    without arithmetic mean value calculation.
    \end{longtable}

    \strut \\
    \textbf{\large Use Case 7}
    
    \begin{longtable}{|p{0.227\linewidth}|p{0.773\linewidth}|}
    \hline
    \textbf{Title} & \textbf{Get Mean Absolute Deviation Value} \\
    \hline
    \endfirsthead
    \hline
    \textbf{Title} & \textbf{Get Mean Absolute Deviation Value} \\
    \hline
    \endhead
    \hline
    \endfoot
    \hline
    \endlastfoot
    Description & Calculate the mean absolute deviation (MAD) from the given
    data set. MAD measures the average absolute difference between each data
    value and the arithmetic mean.
    
    \includegraphics[width=1.71262in,height=0.54336in]{METRICSTICS/media/MAD.png} \\ \hline
    Primary Actor & Technical User \\ \hline
    Preconditions & 1. The system must be running.
    
    2. A valid data set must have been inputted into the system. \\ \hline
    Success Guarantee & The Mean Absolute Deviation value is successfully
    calculated and displayed. \\ \hline
    Main Success Scenario & 1. The system retrieves the sorted data set.
    
    2. The system calculates the mean absolute deviation (MAD) using the
    formula.
    
    3. The system displays the calculated mean absolute deviation value.
    
    4. The user returns to the main menu of operations. \\ \hline
    Extensions & If the data set is empty, an error message indicating that
    the input is empty is displayed, and the user returns to the main menu
    without Mean Absolute Deviation value calculation.
    \end{longtable}
    
    \pagebreak
    
    \strut \\
    \textbf{\large Use Case 8}
    
    \begin{longtable}{|p{0.227\linewidth}|p{0.773\linewidth}|}
    \hline
    \textbf{Title} & \textbf{Get Standard Deviation} \\
    \hline
    \endfirsthead
    \hline
    \textbf{Title} & \textbf{Get Standard Deviation} \\
    \hline
    \endhead
    \hline
    \endfoot
    \hline
    \endlastfoot
    Description & Calculate the standard deviation from the given data set.
    The standard deviation measures the dispersion or spread of data values
    around the arithmetic mean.
    
    \includegraphics[width=1.38351in,height=0.66293in]{METRICSTICS/media/StandardDeviation.png} \\ \hline
    Primary Actor & Technical User \\ \hline
    Preconditions & 1. The system must be running.
    
    2. A valid data set must have been inputted into the system. \\ \hline
    Success Guarantee & The Standard Deviation value is successfully
    calculated and displayed. \\ \hline
    Main Success Scenario & 1. The system retrieves the sorted data set.
    
    2. The system calculates the standard deviation by using the formula.
    
    3. The system displays the calculated standard deviation value.
    
    4. The user returns to the main menu of operations. \\ \hline
    Extensions & If the data set is empty, an error message indicating that
    the input is empty is, and the user returns to the main menu without
    Standard Deviation value calculation.
    \end{longtable}

    \strut \\
    \textbf{\large Use Case 9}
    
    \begin{longtable}{|p{0.227\linewidth}|p{0.773\linewidth}|}
    \hline
    \textbf{Title} & \textbf{Get All Results} \\
    \hline
    \endfirsthead
    \hline
    \textbf{Title} & \textbf{Get All Results} \\
    \hline
    \endhead
    \hline
    \endfoot
    \hline
    \endlastfoot
    Description & Display all calculated descriptive statistics, i.e.,
    minimum, maximum, mode, median, arithmetic mean, mean absolute
    deviation, and standard deviation for the given data set. \\ \hline
    Primary Actor & Technical User \\ \hline
    Preconditions & 1. The system must be running.
    
    2. A valid data set must have been inputted into the system. \\ \hline
    Success Guarantee & The calculated descriptive statistics are
    successfully computed and displayed to the Technical User. \\ \hline
    Main Success Scenario & \begin{minipage}[t]{\linewidth}\raggedright
    1. The system retrieves the sorted data set.
    
    2. The system computes and displays the following descriptive statistics
    to the user:
    
    \begin{itemize}
    \item
      Minimum
    \item
      Maximum
    \item
      Mode
    \item
      Median
    \item
      Arithmetic Mean
    \item
      Mean Absolute Deviation
    \item
      Standard Deviation
    \end{itemize}
    
    3. The system displays the calculated standard deviation value.
    
    4. The user returns to the main menu of operations.
    \end{minipage} \\ \hline
    Extensions & If no valid descriptive statistics are available due to an
    empty data set or invalid result values, an error message indicating the
    same is, and the user returns to the main menu without displaying any
    statistics.
    \end{longtable}
    
    \pagebreak

    \strut \\
    \textbf{\large Use Case 10}
    
    \begin{longtable}{|p{0.227\linewidth}|p{0.773\linewidth}|}
    \hline
    \textbf{Title} & \textbf{Save Session} \\
    \hline
    \endfirsthead
    \hline
    \textbf{Title} & \textbf{Save Session} \\
    \hline
    \endhead
    \hline
    \endfoot
    \hline
    \endlastfoot
    Description & Save the input data reference and descriptive statistics
    to a session file if the user selects the `save session' option. \\ \hline
    Primary Actor & Technical User \\ \hline
    Preconditions & 1. The system must be running.
    
    2. Input file exists. \\ \hline
    Success Guarantee & The input data reference and descriptive statistics
    are saved in the session file. \\ \hline
    Main Success Scenario & 1. The system prompts the user to provide a
    session name.
    
    2. The system adds an entry into the session file containing the session
    name, input data reference and descriptive statistics.
    
    3. The system returns to the main menu. \\ \hline
    Extensions & If the session name already exists, an error message
    indicating that a similar session name already exists is displayed, and
    the user returns to the main menu without saving the session. \\
    \hline
    \end{longtable}

    \strut \\
    \textbf{\large Use Case 11}
    
    \begin{longtable}{|p{0.227\linewidth}|p{0.773\linewidth}|}
    \hline
    \textbf{Title} & \textbf{Load Session} \\
    \hline
    \endfirsthead
    \hline
    \textbf{Title} & \textbf{Load Session} \\
    \hline
    \endhead
    \hline
    \endfoot
    \hline
    \endlastfoot
    Description & Load the input data and descriptive statistics based on
    the session name selected by the user if the user selects the `load
    session' option. \\ \hline
    Primary Actor & Technical User \\ \hline
    Preconditions & 1. The system must be running.
    
    2. Session file and input data file exists. \\ \hline
    Success Guarantee & The input data and calculated descriptive statistics
    are loaded in the system based on user selected session name. \\ \hline
    Main Success Scenario & 1. The system displays a list of available
    session names.
    
    2. The user selects the session name they want to load.
    
    3. The system loads the input data along with the descriptive
    statistics.
    
    4. The system returns to the main menu. \\ \hline
    Extensions & None
    \end{longtable}

    \strut \\
    \textbf{\large Use Case 12}
    
    \begin{longtable}{|p{0.227\linewidth}|p{0.773\linewidth}|}
    \hline
    \textbf{Title} & \textbf{Exit} \\
    \hline
    \endfirsthead
    \hline
    \textbf{Title} & \textbf{Exit} \\
    \hline
    \endhead
    \hline
    \endfoot
    \hline
    \endlastfoot
    Description & Exit the program if the user selects the exit option. \\ \hline
    Primary Actor & Technical User \\ \hline
    Preconditions & 1. The system must be running. \\ \hline
    Success Guarantee & The program terminates successfully. \\ \hline
    Main Success Scenario & 1. The program must end all its processes.
    
    2. The user is no longer able to interact with the program. \\ \hline
    Extensions & None
    \end{longtable}
    