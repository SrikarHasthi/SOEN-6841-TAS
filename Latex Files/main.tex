\documentclass[a4Paper]{article}
\raggedbottom
\usepackage[margin=1in,footskip=.25in]{geometry}
\usepackage{graphicx}
\usepackage{xspace}
\usepackage{lipsum}
\usepackage{abstract}
\renewcommand{\abstractnamefont}{\normalfont\Large\bfseries}
\renewcommand{\abstracttextfont}{\normalsize}

\usepackage{caption}
\usepackage{glossaries}

\usepackage{hyperref}
\urlstyle{same}

\usepackage{enumitem}
\setlist{leftmargin=*}

\begin{document}

\begin{titlepage}
%\setlength{\voffset}{-0.1in}
%\setlength{\headsep}{5pt}
%\setlength{\textheight}{650pt}
%titlepage
%\thispagestyle{empty}
\clearpage
\vspace*{\fill}
\begin{center}
%\begin{minipage}{0.75\linewidth}
    \centering
     
%=====================================================%
%---------------TITLE--------------------------%
%=====================================================%    
      
  {{\Large \textbf{TOPIC ANALYSIS AND SYNTHESIS}\par}}
    \vspace{1cm}% FOR CREATING DESIRED SPACE BETWEEN THE LINES
     
%=====================================================%
%---------------Partial Fulfillment---------------------%
%=====================================================%
    \vspace{0.4cm}
    {
    \textbf{\large Software Project Management (SOEN 6841)} \par}
    \vspace{0.2cm}

%=====================================================%
%---------------Degree---------------------%
%=====================================================%

\vspace{0.5cm}
    {
    \textbf{\Large What personality type fits best into
project management?} \par}
    \vspace{2mm}


    

%=====================================================%
%---------------AUTHOR'S NAME-------------------------%
%=====================================================%
\vspace{5mm}
    {\large by\par}
    \vspace{0.05cm}
    {\small \textbf{Srikar Hasthi}  (40230004)\par}
%    {\Large \textbf{(2013MEZ8469)}\par}
    \vspace{0.9cm}
    
%=====================================================%
%---------------Supervisor Name---------------------%
%=====================================================%    
%    {\Large \textbf{Doctor of Philosophy} \par}
%    \vspace{0.5cm}
    { Under the Supervision of \par}
    {\small \textbf{\textbf{Professor:{ Pankaj Kamthan}}}\par}
    {\small \textbf{\textbf{TA:{ Iymen Abdella}}}\par}
%    {\Large \textbf{(2013MEZ8469)}\par}
     

%=====================================================%
%---------------UNIVERSITY lOGO-----------------------%
%=====================================================%
\includegraphics[width=0.45\linewidth]{TAS/media/Concordia-university-logo.jpg}
%    \rule{0.4\linewidth}{0.15\linewidth}\par

    
%=====================================================%
%---------------DEPARTMENT'S NAME-------------------------%
%=====================================================%

    {\large \textbf{Department of Computer Science and Software Engineering}\par}
    \vspace{0.4cm}
%=====================================================%
%---------------DATE----------------------------------%
%=====================================================%

%\end{minipage}
\end{center}
\vfill % equivalent to \vspace{\fill}
\clearpage
\end{titlepage}


\tableofcontents % Write out the Table of Contents
\pagebreak
\begin{abstract}
The role of personality in project management is a critical factor in ensuring successful project outcomes. The suitability of a project manager's personality type depends on various project-specific factors, such as project size and team experience. Understanding personality types, particularly through models like the Myers-Briggs Type Indicator (MBTI), can shed light on how individuals approach tasks and decision-making. Extroversion is valuable in project management, as it fosters engagement with team members. Project managers inclined towards logical, analytical decision-making tend to excel in managing projects, particularly technical ones. \\

Besides personality, technical expertise matters, but the degree varies with project size. Small technical projects may require a highly technical leader, while larger programs demand leaders who can ensure clear communication and delegate effectively. Effective project managers are detail-oriented, pragmatic, and adept at balancing time, scope, and cost to deliver business value. Their positive and upbeat attitude helps them gain trust from stakeholders, even in challenging times, as they communicate honestly and offer credible strategies for recovery.\\

Personality traits, technical competence, and a positive outlook play crucial roles in project management. The choice of personality type should align with the project's specific needs and characteristics to enhance the chances of project success.
\end{abstract}
\pagebreak
\section{Introduction}
\vspace{1\baselineskip}

\subsection{Motivation}

The motivation for this research is to investigate the relationship between personality type and project management success. Project management is a complex and challenging field, and the success of a project often depends on the ability of the project manager to effectively lead and motivate a team.\\

There is a growing body of research on the relationship between personality type and job performance in general, but there is less research on the relationship between personality type and project management success specifically. The need to explore how specific personality traits align with various project requirements is vital, as it directly impacts project outcomes and team dynamics. This is a gap in the literature that this research aims to address.

\subsection{Problem Statement}
The specific problem statement being investigated in this research is:\\

How does personality type influence project management success?\\

The investigation focuses on the precise examination of how different personality types of influence project management outcomes. With projects becoming more intricate, diverse, and team-oriented, it is imperative to identify the specific traits that lead to successful project delivery. The problem lies in the challenge of pinpointing these traits accurately and understanding their nuanced impact on project management processes. Addressing this challenge is crucial for businesses and organizations striving to optimize their project teams and ensure efficient project execution.

\subsection{Objectives}
The objectives of this research are to:
\begin{enumerate}
    \item Identify the key personality traits that are associated with project management success.
    \item Develop a model that can predict project management success based on personality type.
    \item Provide recommendations for how project managers can improve their success by developing their key personality traits.
\end{enumerate}
This research will benefit both project managers and organizations that employ project managers. By understanding the relationship between personality type and project management success, project managers can develop their key personality traits and organizations can select the right project managers for their projects. The ultimate goal of this research is to improve project teams' productivity and efficiency in a variety of industries, which will benefit people and companies alike.

\pagebreak

\section{Background Material}

In recent years, the field of project management has evolved significantly, with a growing emphasis on the human element within projects. Understanding the diverse range of personalities within project teams has become crucial for effective leadership and collaboration. Several established psychological models, such as the Myers-Briggs Type Indicator (MBTI), have been widely used to categorize individuals into distinct personality types, shedding light on their behaviors, preferences, and decision-making processes.\\

Research in organizational psychology has explored the impact of personality traits on team dynamics, communication styles, and leadership effectiveness. Studies have shown that extroverted individuals tend to excel in roles requiring social interaction and team collaboration, making them potentially effective project managers in people-centric projects. Conversely, introverted individuals might thrive in roles demanding focused attention to detail and independent problem-solving.[3]\\

Personality type is a set of relatively stable personality traits that influence how people think, feel, and behave. There are many different personality type models, but the Myers-Briggs Type Indicator (MBTI) is one of the most popular. The MBTI classifies people into one of 16 personality types based on four dichotomies:\\

\textbf{Introversion (I) vs. Extroversion (E)}: Introverts prefer to focus on their inner world of thoughts and feelings, while extroverts prefer to focus on the outer world of people and activities.\\

\textbf{Sensing (S) vs. Intuition (N)}: Sensors prefer to focus on concrete, observable facts, while intuitives prefer to focus on abstract concepts and patterns.\\

\textbf{Thinking (T) vs. Feeling (F)}: Thinkers prefer to make decisions based on logic and reason, while feelers prefer to make decisions based on emotions and values.\\

\textbf{Judging (J) vs. Perceiving (P)}: Judgers prefer to have things planned and decided, while perceivers prefer to be flexible and spontaneous.\\

Research on the relationship between personality type and job performance has found that individuals who are better matched to their jobs are more likely to be successful in their jobs. This is because individuals who are better matched to their jobs are more likely to be satisfied with their jobs, which leads to increased motivation and performance.[4]\\

Additionally, studies in decision-making processes have highlighted the significance of personality traits in determining whether individuals rely on analytical, data-driven approaches or intuitive, experiential judgments. Project managers who base their decisions on logical analysis are often better equipped to handle complex, technical projects, ensuring that tasks are approached systematically and objectives are met efficiently.\\

By delving into these existing studies and theories, this investigation aims to build upon the established knowledge in the field of organizational psychology and project management. By synthesizing this background material with empirical research, the study aims to provide a comprehensive understanding of how specific personality types contribute to successful project management, offering practical implications for professionals and organizations involved in diverse projects.


\pagebreak
\section{References}

    \begin{enumerate}[label={[\arabic*]}]
        \item Judge, T. A., Bono, J. E., Ilies, R., \& Gerhardt, M. W. (2002). The conscientiousness-performance relationship: A meta-analysis. Personnel Psychology, 55(3), 797-82 \url{http://www.timothy-judge.com/Judge,%20Bono,%20Ilies,%20&%20Gerhardt.pdf}

        \item Myers, I. B., \& Briggs Myers, K. (1962). Manual: The Myers-Briggs type indicator. Princeton, NJ: Educational Testing Service \url{https://www.semanticscholar.org/paper/The-Myers-Briggs-Type-Indicator%3A-Manual-(1962).-Myers/d39f44fd94ab4e73a826aa1e4c0fcbc49377838d}

        \item Cohen, Y., Ornoy, H., \& Keren, B. (2013). Personality types of project managers and their success: A field survey. Psychology \& Marketing, 30(11), 927-935. \url{https://www.researchgate.net/publication/263596981_MBTI_Personality_Types_of_Project_Managers_and_Their_Success_A_Field_Survey}

        \item Hussain, A., Jamil, M., Farooq, M. U., Asim, M., Rafique, M. Z., \& Pruncu, C. (2021). Project Managers’ Personality and Project Success: Moderating Role of External Environmental Factors. Sustainability, 13(16), 9477.\url{https://www.researchgate.net/publication/354094465_Project_Managers'_Personality_and_Project_Success_Moderating_Role_of_External_Environmental_Factors}

        \item Creswell, J. W. (2014). Research design: Qualitative, quantitative, and mixed methods approaches (4th ed.). Thousand Oaks, CA: Sage Publications. \url{https://www.ucg.ac.me/skladiste/blog_609332/objava_105202/fajlovi/Creswell.pdf}

        \item Braun, V., \& Clarke, V. (2006). Using thematic analysis in psychology. Qualitative Research in Psychology, 3(2), 77-101.\url{https://www.researchgate.net/publication/235356393_Using_thematic_analysis_in_psychology}
  
    \end{enumerate}
\pagebreak
\end{document}